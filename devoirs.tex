\documentclass[a4paper, 11pt]{article}
\usepackage[utf8]{inputenc}
\usepackage[T1]{fontenc}
\usepackage[french]{babel}

\usepackage{lmodern}
\usepackage{textcomp}
\usepackage{ifthen} \usepackage{amsmath} \usepackage{amsfonts} \usepackage{amssymb} \usepackage{graphicx}
\usepackage{enumitem}
\usepackage{multicol}
\usepackage[pagenumber]{polytechnique}
\usepackage[pdftex=true,
			colorlinks=true,
			linkcolor=black,
			filecolor=red,
			urlcolor=blue,
			bookmarks=true,
			bookmarksopen=true]{hyperref}

\title{Questions métaphysiques}
\author{\longpie \\
		\longgus \\
		\longcle \\
		\longfel \\
		\longale \\
		\longfra \\
		\longpal \\
		\longgui}
\subtitle{Projet Scientifique Collectif}
\date{\today}

\begin{document}
%\pagestyle{} %ou plain, headings, empty
\maketitle
\tableofcontents

\clearpage

%================DEBUT DOCUMENT=====================
\annee{2015}
	\mois{Octobre}5
		\semaine
		
		\obj{Chaque personne devra étudier un protocole, et préparer de quoi le présenter au reste du groupe.}
		
		\task{tout le groupe}{
		\begin{itemize}
			\item \'Etudier le protocole qui a été attribué.
			\item Rédiger une fiche \Acc{courte mais complète} expliquant les points importants du protocole.
			\item Présenter le protocole au groupe. Tous les moyens sont bons (tableau, diaporama, etc.), mais cela ne sert à rien de faire quelque chose de trop long ou compliqué. Environ 5 minutes de présentation, quoi.
		\end{itemize}
		}
		
		\partie{Attribution des protocoles}{
		\begin{description}
			\item[TCP] 
			\item[UDP] 
			\item[DNS] 
			\item[ARP] 
			\item[HTTP] 
			\item[FTP] 
		\end{description}
		}
		
		\semaine
		
		
		\semaine
		
		
		\semaine
		
		
	\mois{Novembre}2
		\semaine
\end{document}